\documentclass{article}
\usepackage[top=0.75in, bottom=0.75in, left=1.25in, right=1in]{geometry} % formatage
\usepackage[frenchb]{babel}
\usepackage[utf8]{inputenc} 
\usepackage{amsmath} % pour utiliser des maths de base 
\usepackage{amssymb} % pour faire \mathcal{}=>des lettres ''cursives''
\usepackage{amsthm} % La petite boîte de fin de preuve
\usepackage{graphicx} % pour importer des images...http://www.tex.ac.uk/cgi-bin/texfaq2html?label=figurehere
\usepackage{titlesec} % automatique, pour faire des sous-titres moins laids
%\usepackage{cancel}
\usepackage[procnames]{listings}
\usepackage[T1]{fontenc}        % http://tex.stackexchange.com/questions/11897/draw-a-diagonal-arrow-across-an-expression-in-a-formula-to-show-that-it-vanishes%
\usepackage[squaren]{SIunits}
\usepackage{subcaption} % Avoir plusieurs sous-figures (graphiques) dans une figures et pouvoire les étiqueter
\usepackage{color}
\usepackage{lipsum}
\usepackage{caption}
\usepackage{enumitem} % Permet d'avoir plus de flexibilité dans les enumerations.
\usepackage{wasysym} 
\usepackage{braket}
\usepackage{mathtools}
\usepackage{multirow} % Fusionner des lignes dans un tableau
\usepackage{mathrsfs} % Faire le symbole de la transformée de Laplace
\usepackage{bbm}
\usepackage{array}
\usepackage{diagbox} % diagonale dans les tableaux
\usepackage{dsfont} % Faire des belles indicatrices                         
\usepackage{float} % placer les tableaux et images où tu veux
\usepackage{listings}
\usepackage[utf8]{inputenc}
\usepackage{comment}
\usepackage{pst-node}
\usepackage{fancyvrb} % Les varbatims gardent l'indentation
\usepackage{enumitem}
\usepackage{breakcites} % Faire en sorte que les citations ne sortent pas dans la marge
\usepackage{graphicx} % Insérer des graphiques
\usepackage{pgfplots}
\usepackage{hyperref} % Faire des hyperliens
\usepackage{verbatim} % Inclure un fichier .text en verbatim
\usepackage{xcolor}
\pgfplotsset{width=10cm, compat=1.9}

% Changer la couleur des hyperliens
\hypersetup{colorlinks = true,
	allcolors  = blue, % default color = black
	%	citecolor  = black
}  


% redefine \VerbatimInput
\RecustomVerbatimCommand{\VerbatimInput}{VerbatimInput}%
{fontsize=\footnotesize,
	%
	frame=lines,  % top and bottom rule only
	framesep=2em, % separation between frame and text
	rulecolor=\color{Gray},
	%
	label=\fbox{\color{Black}data.txt},
	labelposition=topline,
	%
	commandchars=\|\(\), % escape character and argument delimiters for
	% commands within the verbatim
	commentchar=*        % comment character
}


\newcommand{\RomanNumeralCaps}[1]
    {\MakeUppercase{\romannumeral #1}}

\newtheorem{lemme}{Lemme}
\newtheorem{preuve}{Preuve}
\newtheorem{code}{Code informatique}
\newtheorem{exemple}{Exemple}
\newtheorem{scenario}{Scénario}
\newtheorem{algo}{Algorithme}
\newtheorem{definition}{Définition}
\newtheorem{proposition}{Proposition}
\newtheorem{propriete}{Propriété}
\newtheorem{test_hypothese}{Teste d'hypothèse}

\begin{document}
	\renewcommand{\tablename}{Tableau}
	\renewcommand{\figurename}{Illustration}
	\renewcommand{\P}{\mathbb{P}}
	
	\begin{titlepage}
		\centering % Centre everything on the title page
		
		\scshape % Use small caps for all text on the title page
		
		\vspace*{7\baselineskip} % White space at the top of the page
		
		%------------------------------------------------
		%	Title
		%------------------------------------------------
		
		\rule{\textwidth}{1.6pt}\vspace*{-\baselineskip}\vspace*{2pt} % Thick horizontal rule
		\rule{\textwidth}{0.4pt} % Thin horizontal rule
		
		\vspace{0.75\baselineskip} % Whitespace above the title
		{\LARGE Travail pratique 2\\} % Title
%		\vspace{0.75\baselineskip}
		\vspace{0.75\baselineskip} % Whitespace below the title
		
		\rule{\textwidth}{0.4pt}\vspace*{-\baselineskip}\vspace{3.2pt} % Thin horizontal rule
		\rule{\textwidth}{1.6pt} % Thick horizontal rule
		
		\vspace{4\baselineskip} % Whitespace after the title block
		
		%------------------------------------------------
		%	Subtitle
		%------------------------------------------------
		
		Travail présenté à \\
		{\scshape\Large M. Thierry Duchesne\\}
		
		\vspace*{4\baselineskip}
		
		Dans le cadre du cours\\
		{\scshape\Large Théorie et applications des méthodes de régression \\ STT-7125}
		
		% Subtitle or further description
		
		\vspace*{4\baselineskip} % Whitespace under the subtitle
		
		%------------------------------------------------
		%	Editor(s)
		%------------------------------------------------
		
		Réalisé par l'équipe 21:\\
		{\scshape\Large Alexandre Lepage\\
		\& Amedeo Zito} % Editor list
		
		\vspace*{5\baselineskip}
		
		le 17 décembre 2020
		
		\vspace{0.4\baselineskip} % Whitespace below the editor list
		
		\vfill % Whitespace between editor names and publisher logo
		
		%------------------------------------------------
		%	Publisher
		%------------------------------------------------
		
		\includegraphics[height=1.2cm]{UL_P.pdf}\\
		
		Faculté des sciences et de génie\\
		École d'actuariat\\
		Université Laval
		
		\vspace*{3\baselineskip}
		
	\end{titlepage}
	
%	\pagenumbering{Roman} % Pagination en chiffres romains
%	\setcounter{page}{0} 
%	
%	\newpage
%	\strut % Page blanche
%	\newpage
%	
%	\tableofcontents
%	\renewcommand{\listfigurename}{Liste des illustrations}
%	\newpage
%	
%	\listoffigures
%	\listoftables
%	\newpage
%	
%	\pagenumbering{arabic} % Pagination en chiffres normaux
%	\setcounter{page}{1}


\section{Introduction}


\section{Question 1}\label{sect_qst1}


	

	
	
\end{document}